\section{Understanding Cardinality and Countability}\cite{understandinganalysis}

One theorem, discovered in 500 BC, had lasting impact on the branch of mathematics for the next 2000 years! It was none other than the discovery of $\sqrt{2}$ by Pythagoras. \\

In 500 BC, Greeks had understanding of the relationship between the geometric length and arithmetic numbers. It was known back then that - given any two line segments $AB$ and $CD$, we can represent $CD$ as some fractional multiple of $AB$. At that time, Pythagoras discovered the length of hypotenuse of a right isosceles triangle with sides 1 unit long to be $\sqrt{2}$ units. Greeks for centuries couldn't understand the notion of a \textit{number} because this length cannot be represented as a fractional number - as proved by Pythagoras.

\begin{theorem}
    There is no rational number whose square is 2.
\end{theorem}

\begin{proof}
Let us assume, for the sake of contradiction, that $\sqrt{2}$ is rational. Then, we can express $\sqrt{2}$ as a fraction of two integers $a$ and $b$, where $a$ and $b$ have no common factors other than $1$ (i.e., the fraction is in its simplest form):
\[
\sqrt{2} = \frac{a}{b}
\]
Squaring both sides, we get:
\[
2 = \frac{a^2}{b^2}
\]
Multiplying both sides by $b^2$, we obtain:
\[
2b^2 = a^2
\]
This equation implies that $a^2$ is an even number (since it is equal to $2b^2$, which is even). Therefore, $a$ must also be an even number (because the square of an odd number is odd).

Let $a = 2k$ for some integer $k$. Substituting this into the equation $2b^2 = a^2$, we get:
\[
2b^2 = (2k)^2 = 4k^2
\]
Dividing both sides by $2$, we get:
\[
b^2 = 2k^2
\]
This implies that $b^2$ is also even, and hence $b$ must be even. \\

Therefore, both $a$ and $b$ are even, which contradicts our assumption that $a$ and $b$ have no common factors other than $1$ (since both being even means they are divisible by $2$). \\

Thus, our original assumption that $\sqrt{2}$ is rational must be false. Hence, $\sqrt{2}$ is irrational.
\end{proof}

With our existing knowledge about the numbers, let's try to logically extend the system of numbers from the naturals to the irrationals. The most intuitive set of numbers is the set of counting numbers, i.e. the natural numbers $\N$.

\begin{center}
    $\N = \{1, 2, 3, ...\}$
\end{center}

By focusing on the set of natural numbers \( \mathbb{N} \), we can handle addition without any issues. However, to introduce the concept of subtraction, we need to extend our number system to the set of integers 
\( \mathbb{Z} = \{ \dots, -3, -2, -1, 0, 1, 2, 3, \dots \} \), which includes the additive identity (zero) and additive inverses. Next, we turn to multiplication and division. The number 1 serves as the multiplicative identity, but to define division, we require multiplicative inverses. This leads us to extend our system further to the rational numbers
\[
\mathbb{Q} = \left\{ \frac{p}{q} \ \middle|\ p, q \in \mathbb{Z},\ q \neq 0 \right\}
\]
which includes all fractions. \\

The properties discussed essentially define what is known as a field. Recall how $\mathbb{Q}$ is a field. A field is any set where addition and multiplication are well-defined operations that satisfy commutativity, associativity, and the distributive law: \( a(b + c) = ab + ac \). Additionally, there must be an additive identity, and every element must have an additive inverse. Similarly, there must be a multiplicative identity, and multiplicative inverses must exist for all nonzero elements. Neither \( \mathbb{Z} \) nor \( \mathbb{N} \) is a field. \\

The set $\mathbb{Q}$ has a natural ordering. For any two rational numbers $r$ and $s$, exactly one of the following holds: $r < s$, $r = s$, or $r > s$. This order is transitive, meaning if $r < s$ and $s < t$, then $r < t$. This allows us to visualize $\mathbb{Q}$ as arranged along a number line from left to right. Unlike $\mathbb{Z}$, there are no gaps, since between any two rational numbers $r < s$, the rational number $\frac{r+s}{2}$ lies between them, showing that $\mathbb{Q}$ is dense. \\

While the field properties of $\mathbb{Q}$ enable us to perform addition, subtraction, multiplication, and division, there are still limitations. By Theorem 1.1.1, not all numbers, such as square roots, can be expressed as rationals. This issue is deeper than just square roots. We can approximate irrational numbers like $\sqrt{2}$ using rational numbers (for example, $1.4142 \approx 1.999396$), but despite better approximations, we realize that there are \textit{gaps} in $\mathbb{Q}$, such as at $\sqrt{2}$. Similar gaps exist at $\sqrt{3}$, $\sqrt{5}$, and other points. This dilemma, faced by the ancient Greeks, reveals the need for a more complete number system. Thus, we extend $\mathbb{Q}$ to the real numbers $\mathbb{R}$, creating the chain $\mathbb{N} \subseteq \mathbb{Z} \subseteq \mathbb{Q} \subseteq \mathbb{R}$. \\

The process of constructing $\mathbb{R}$ from $\mathbb{Q}$ is quite intricate and is explored later in this section. For now, a simplified view is that $\mathbb{R}$ is formed by filling the gaps in $\mathbb{Q}$. Whenever there is a missing value, a new irrational number is introduced and placed within the existing order of $\mathbb{Q}$. The real numbers consist of both these irrational numbers and the familiar rational numbers. But what characteristics do the irrational numbers possess? 

\begin{itemize}
    \item How do the rational and irrational numbers interrelate? Is there any symmetry between them, or can we argue that one type is more prevalent than the other? 
    \item So far, we have seen examples of irrational numbers through square roots. As expected, other roots like $\sqrt[3]{2}$ or $\sqrt[5]{3}$ are typically irrational as well. Can all irrational numbers be represented as algebraic combinations of roots and rational numbers, or are there irrationals beyond this form?
\end{itemize}

Answering these questions is not trivial and hence we will take a step-by-step approach - developing concepts from the preliminaries. 

\subsection{Set Theory}\cite{probabilityfoundations}

A set is understood as a collection of distinct objects, known as elements. An object either belongs to the set or it does not, which makes the set well-defined. Sets can be described in two ways:

\begin{enumerate}
    \item \textbf{Extensional Definition:} This involves explicitly listing all elements of the set within curly braces. For example, the set of natural numbers from 1 to 5 can be written as \( A = \{1, 2, 3, 4, 5\} \).
    \item \textbf{Intensional Definition:} In this method, a set is described by a property that all its elements satisfy. This is also known as set-builder notation. The set \( A \) from above can be expressed as \( A = \{x \mid x \leq 5, x \in \mathbb{N}\} \). In general, a set \( C \) can be written as \( C = \{x \mid P(x)\} \), where \( P(x) \) represents some property.
\end{enumerate}

We now introduce the concept of a subset and use it to define when two sets are equal.

\begin{definition} 
    \begin{enumerate}
        \item A set \( A \) is called a subset of another set \( B \) if every element of \( A \) is also an element of \( B \). This is written as \( A \subseteq B \), and in this case, \( B \) is called a superset of \( A \). 
        \item \( A \) is a proper subset of \( B \) (denoted \( A \subset B \)) if \( A \subseteq B \) and there is at least one element in \( B \) that is not in \( A \).  
        \item Two sets \( A \) and \( B \) are equal if \( A \subseteq B \) and \( B \subseteq A \), meaning both sets contain exactly the same elements.\\   
    \end{enumerate}
\end{definition}

\subsubsection{Operations on Sets}
\vspace{5pt}

\textbf{1. Complement}

\begin{definition}
    For a given set \(A\), its complement is denoted as \(A^c\) and is defined as \(\{x \mid x \notin A, x \in U\}\), where \(U\) represents the universal set containing \(A\). 
\end{definition}

The complement is always understood within the context of a larger set \(U\). \\

\textbf{2. Union and Intersection} \\

Let \(I\) be an index set, and consider a collection of sets \(\{A_i \mid i \in I\}\).

\begin{definition}
    The union of the sets \(\{A_i \mid i \in I\}\) is defined as:
\[
\bigcup_{i \in I} A_i = \{x \mid x \in A_j \text{ for some } j \in I\}
\]
\end{definition}

In other words, the union consists of all elements that belong to at least one of the sets \(A_i\). 

\begin{definition}
    The intersection of the sets \(\{A_i \mid i \in I\}\) is defined as:
\[
\bigcap_{i \in I} A_i = \{x \mid x \in A_j \text{ for every } j \in I\}
\]
\end{definition}

In simpler terms, the intersection consists of elements common to all sets \(A_i\). \\

\textbf{\textit{Note}}: When the index set \(I\) is finite, say \(I = \{1, 2, 3\}\), the union matches the intuitive understanding, i.e., \(\bigcup_{i=1}^{3} A_i = A_1 \cup A_2 \cup A_3\). However, this interpretation does not apply when \(I\) is infinite, such as \(I = \mathbb{N}\). In this case, \(\bigcup_{i=1}^{\infty} A_i\) cannot be thought of as a sequential process of unions. Instead, it should be viewed as described in Definition 1.13: the set of elements in at least one \(A_i\), where \(i \in \mathbb{N}\). \\

The following identities related to unions and intersections can be easily derived:

\begin{equation}
\bigcap_{i \in I} (A_i \cup B) = \bigcap_{i \in I} A_i \cup B 
\end{equation}

\begin{equation}
\bigcup_{i \in I} (A_i \cap B) = \bigcup_{i \in I} A_i \cap B 
\end{equation}

\textit{De Morgan's Laws}, which explain the relationship between unions, intersections, and complements, are particularly useful:

\begin{theorem}
\begin{equation}
            \left(\bigcap_{i \in I} A_i\right)^c = \bigcup_{i \in I} A_i^c
        \end{equation}
\begin{equation}
            \left(\bigcup_{i \in I} A_i\right)^c = \bigcap_{i \in I} A_i^c
        \end{equation}
\end{theorem}

\begin{proof}
    
    Let \( x \in \left( \bigcap_{i \in I} A_i \right)^c \). By the definition of complement, this implies:
    \[
    x \notin \bigcap_{i \in I} A_i
    \]
    which means that there exists some \( j \in I \) such that \( x \notin A_j \). Therefore, \( x \in A_j^c \) for some \( j \in I \). This implies:
    \[
    x \in \bigcup_{i \in I} A_i^c
    \]
    Hence, we have shown that:
    \[
    x \in \left( \bigcap_{i \in I} A_i \right)^c \implies x \in \bigcup_{i \in I} A_i^c
    \]
    
    Now, for the reverse direction, assume \( x \in \bigcup_{i \in I} A_i^c \). This means there exists some \( j \in I \) such that \( x \in A_j^c \), or equivalently, \( x \notin A_j \). Therefore, \( x \notin \bigcap_{i \in I} A_i \), which implies:
    \[
    x \in \left( \bigcap_{i \in I} A_i \right)^c
    \]
    
    Thus, we have proven that:
    \[
    x \in \left( \bigcap_{i \in I} A_i \right)^c \iff x \in \bigcup_{i \in I} A_i^c
    \]

    This completes the proof of the first law.

    \vspace{10pt}
    
    Let \( x \in \left( \bigcup_{i \in I} A_i \right)^c \). By the definition of complement, this implies:
    \[
    x \notin \bigcup_{i \in I} A_i
    \]
    which means that \( x \notin A_i \) for every \( i \in I \). Therefore, \( x \in A_i^c \) for all \( i \in I \). This implies:
    \[
    x \in \bigcap_{i \in I} A_i^c
    \]
    Hence, we have shown that:
    \[
    x \in \left( \bigcup_{i \in I} A_i \right)^c \implies x \in \bigcap_{i \in I} A_i^c
    \]
    
    Now, for the reverse direction, assume \( x \in \bigcap_{i \in I} A_i^c \). This means \( x \in A_i^c \) for every \( i \in I \), or equivalently, \( x \notin A_i \) for all \( i \in I \). Therefore, \( x \notin \bigcup_{i \in I} A_i \), which implies:
    \[
    x \in \left( \bigcup_{i \in I} A_i \right)^c
    \]
    
    Thus, we have proven that:
    \[
    x \in \left( \bigcup_{i \in I} A_i \right)^c \iff x \in \bigcap_{i \in I} A_i^c
    \]

    This completes the proof of the second law.
\end{proof}

\textbf{3. Relative Complement}

\begin{definition}
    The relative complement of \(B\) in \(A\) is defined as:
\[
A \setminus B = \{x \mid x \in A, x \notin B\} = A \cap B^c
\]
Similarly, the relative complement of \(A\) in \(B\) is \(B \setminus A = \{x \mid x \in B, x \notin A\} = B \cap A^c\).
\end{definition}

\textbf{4. Cartesian Product} \\

A Cartesian product constructs a set from multiple sets by pairing their elements.

\begin{definition}
    The Cartesian product of two sets \(A\) and \(B\) is defined as:
\[
A \times B = \{(x, y) \mid x \in A, y \in B\}
\]
\end{definition}

This represents the set of all ordered pairs where the first element comes from \(A\) and the second from \(B\). For instance, if \(A = \{1, 2\}\) and \(B = \{a\}\), then \(A \times B = \{(1, a), (2, a)\}\) and \(B \times A = \{(a, 1), (a, 2)\}\). Clearly, the Cartesian product is not commutative. \\

For \(n\) sets \(A_1, A_2, \ldots, A_n\), their Cartesian product is:
\[
A_1 \times A_2 \times \cdots \times A_n = \{(a_1, a_2, \ldots, a_n) \mid a_i \in A_i\}
\]


If all the sets are identical, then we have:
\[
A^n = \{(a_1, a_2, \ldots, a_n) \mid a_i \in A\}
\]

\textbf{5. Power Set}

\begin{definition}
    The power set of a set \(A\), denoted by \(\mathcal{P}(A)\) or \(2^A\), is the set of all subsets of \(A\), including the empty set and \(A\) itself. 
\end{definition}

For example, if \(A = \{1, 2\}\), then:
\[
\mathcal{P}(A) = \{\emptyset, \{1\}, \{2\}, \{1, 2\}\}
\]

\subsection{Functions}

Now that we know enough about sets, it's time to create mappings between two sets using \textit{functions}.

\begin{definition}
    A function $f$ from a set $A$ to a set $B$ is a subset of the Cartesian product $(A \times B)$ such that each element in $A$ appears as the first component in exactly one ordered pair in the subset. Essentially, this means that every element in set $A$ is mapped to a unique element in set $B$, often denoted as $f: A \to B$. The set $A$ is called the domain, and set $B$ is the codomain.
\end{definition}

If an element $b \in B$ is associated with an element $a \in A$, $b$ is called the image of $a$, while $a$ is known as the argument or pre-image of $b$. In this case, we say that $f$ maps $a$ to $b$ and write it as $b = f(a)$. The range of a function is the set of all images of elements in the domain, which forms a subset (not necessarily proper) of the codomain. \\

Functions are classified into: 
\begin{enumerate}
    \item \textbf{Injective (One-to-One):} A function is injective if $a \neq b \implies f(a) \neq f(b)$ for all $a, b \in \text{domain}(f)$. For example, the function $f: \mathbb{N} \to \mathbb{R}$ defined by $f(x) = x$ for all $x \in \mathbb{N}$ is injective.
    \item \textbf{Serjective (Onto):} A function is surjective if for every $b \in \text{codomain}(f)$, there exists an $a \in \text{domain}(f)$ such that $f(a) = b$. 
\end{enumerate}

Examples of surjective functions include:
\begin{itemize}
    \item Let $A = \{1, 2, 3\}$ and $B = \{0, 1\}$. The function $g: A \to B$ defined by $g(1) = 0$, $g(2) = 0$, and $g(3) = 1$ is surjective.
    \item The function $h: \mathbb{R} \to \mathbb{R}$ defined by $h(x) = x + 1$ for all $x \in \mathbb{R}$ is also surjective.  
\end{itemize}

A function that is both injective and surjective is called \textit{bijective}. The function $h$ mentioned above is also bijective. In a bijective function, an \textit{inverse function} can be defined as the mapping is unique and covers the entire codomain.

\subsection{The Axiom of Completeness}

What exactly is a \textit{real number?} We got as far as saying that the set $\R$ of real numbers is an extension of the rational numbers $\Q$ in which there are no holes or gaps. We are going to improve this definition. 

Let $S$ be a non-empty subset of real numbers $\mathbb{R}$.

\begin{definition}
    The \textit{infimum} (inf) of $S$, denoted as $\inf S$, is the greatest lower bound of $S$. That is, $\inf S = \alpha$ if:
\begin{itemize}
    \item $\alpha \leq x$ for all $x \in S$ (i.e., $\alpha$ is a lower bound of $S$), and
    \item for every $\epsilon > 0$, there exists some $x \in S$ such that $\alpha + \epsilon > x$ (i.e., $\alpha$ is the greatest such lower bound).
\end{itemize}
\end{definition}

Similarly, 

\begin{definition}
    The \textit{supremum} (sup) of $S$, denoted as $\sup S$, is the least upper bound of $S$. That is, $\sup S = \beta$ if:
\begin{itemize}
    \item $\beta \geq x$ for all $x \in S$ (i.e., $\beta$ is an upper bound of $S$), and
    \item for every $\epsilon > 0$, there exists some $x \in S$ such that $\beta - \epsilon < x$ (i.e., $\beta$ is the least such upper bound).
\end{itemize}
\end{definition}

\begin{definition}
    The Axiom of Completeness states that every non-empty subset of real numbers that is bounded above has a supremum in $\mathbb{R}$. Similarly, every non-empty subset of real numbers that is bounded below has an infimum in $\mathbb{R}$.
\end{definition}

In simpler terms, this axiom guarantees that in the set of real numbers, there are no "gaps" — every set that is bounded from above or below has a least upper bound or a greatest lower bound, respectively. But these are different than the \textit{minimum and maximum}. Consider the following example for clarity.\\

\begin{example}
    Consider the set $S = (0, 1)$, the open interval between 0 and 1.
\begin{itemize}
    \item The \textit{infimum} of $S$ is $\inf S = 0$, as 0 is the greatest number less than or equal to all elements of $S$. However, since 0 is not an element of $S$, it is \textbf{not} the minimum.
    \item The \textit{supremum} of $S$ is $\sup S = 1$, as 1 is the least number greater than or equal to all elements of $S$. However, since 1 is not an element of $S$, it is \textbf{not} the maximum.
\end{itemize}

Now consider the set $T = [0, 1]$, the closed interval between 0 and 1.
\begin{itemize}
    \item The \textit{minimum} of $T$ is 0, as 0 is the smallest element in $T$.
    \item The \textit{maximum} of $T$ is 1, as 1 is the largest element in $T$.
\end{itemize}

Thus, for an open interval like $(0, 1)$, the infimum and supremum exist but the minimum and maximum do not. In contrast, for a closed interval like $[0, 1]$, the infimum is equal to the minimum, and the supremum is equal to the maximum.
\end{example}

Because of this axiom, we can say that \textit{the real line contains no gaps}. \\

\begin{theorem}
    The real line contains no gaps.
\end{theorem}

\begin{proof}

Let \( I_n = [a_n, b_n] \) be a sequence of closed intervals such that: \\

1. Each interval \( I_n \) is nested: 
   \[
   I_{n+1} \subseteq I_n \quad \text{for all } n \in \mathbb{N}
   \]
2. The lengths of the intervals tend to zero:
   \[
   b_n - a_n \to 0 \quad \text{as } n \to \infty
   \]

Let \( x \) be a point in the intersection $\bigcap_{n=1}^{\infty} I_n $. Since \( x \) lies in every \( I_n \), we have:
\[
a_n \leq x \leq b_n \quad \text{for all } n \in \mathbb{N}
\]

By the Axiom of Completeness, every bounded set of real numbers has a least upper bound (supremum) and a greatest lower bound (infimum). The sequences \( (a_n) \) and \( (b_n) \) are bounded, and thus we define:
\[
L = \liminf_{n \to \infty} a_n \quad \text{and} \quad U = \limsup_{n \to \infty} b_n
\]

Since \( b_n - a_n \to 0 \), we have:
\[
U - L = \lim_{n \to \infty} (b_n - a_n) = 0
\]
Thus, \( L = U \). Therefore, there exists a unique limit \( c \) such that:
\[
c = L = U
\]

This means that every nested sequence of closed intervals must converge to a point in \( \mathbb{R} \), implying that there are no gaps in the real line.
    
\end{proof}

If there are no gaps in the real line, it also implies that there exists a real number whose square is 2. 

\begin{theorem}
    There exists a real number whose square is 2.
\end{theorem}

\begin{proof}
    Consider the set 
   \[
   S = \{ x \in \mathbb{R} \mid x^2 < 2 \}.
   \]

   By the Axiom of Completeness, since \( S \) is non-empty and bounded above, there exists a least upper bound \( c = \sup S \).

    \textbf{Show \( c^2 \leq 2 \)}: Suppose \( c^2 > 2 \). Then, there exists some \( \epsilon > 0 \) such that \( c^2 = 2 + \epsilon \). Since \( c \) is the least upper bound of \( S \), we can find \( x \in S \) such that \( c > x \). This implies \( x^2 < 2 \). We can choose \( x \) close enough to \( c \) such that 
     \[
     x^2 > c^2 - \epsilon = 2,
     \]
     contradicting \( x \in S \). Thus, we conclude \( c^2 \leq 2 \).

   \textbf{Show \( c^2 \geq 2 \)}: Suppose \( c^2 < 2 \). Then there exists some \( \epsilon > 0 \) such that \( c^2 = 2 - \epsilon \). Because \( c \) is the least upper bound, there exists \( y \in \mathbb{R} \) such that \( c < y < c + \delta \) for some small \( \delta > 0 \). This implies 
     \[
     y^2 > c^2 = 2 - \epsilon,
     \]
     which can be made greater than \( 2 \) for sufficiently small \( \delta \), contradicting the upper bound property of \( c \). Thus, we conclude \( c^2 \geq 2 \). \\

     Since \( c^2 \leq 2 \) and \( c^2 \geq 2 \), we have 
   \[
   c^2 = 2.
   \]

Thus, there exists a real number \( c \) such that \( c^2 = 2 \).
\end{proof}

\subsection{Cardinality and Countability}

In informal language, the cardinality of a set refers to the quantity of elements within that set. To compare the cardinalities of two finite sets \(A\) and \(B\), one can simply count the elements in each set and determine whether they have the same cardinality or if one set contains more elements than the other. However, when comparing sets with infinitely many elements (for instance, \(\mathbb{N}\) versus \(\mathbb{Q}\)), this basic method is inadequate. In the late nineteenth century, Georg Cantor proposed a more sophisticated approach for comparing cardinalities based on the types of functions that can be defined from one set to another.

\begin{definition}
    \begin{enumerate}
        \item Two sets \(A\) and \(B\) are considered equicardinal (denoted \(|A| = |B|\)) if there exists a bijective function from \(A\) to \(B\).
        \item Set \(B\) has cardinality greater than or equal to that of \(A\) (denoted \(|B| \geq |A|\)) if there is an injective function from \(A\) to \(B\).
        \item Set \(B\) has cardinality strictly greater than that of \(A\) (denoted \(|B| > |A|\)) if there exists an injective function but no bijective function from \(A\) to \(B\).
    \end{enumerate}
\end{definition}

Based on these definitions, the concept of countability of a set is defined as follows:

\begin{definition}
    A set \(E\) is said to be countably infinite if it is equicardinal with \(\mathbb{N}\). A set is classified as \textbf{countable} if it is either finite or countably infinite. \\
\end{definition}


\begin{example}
    The set of even numbers is equicardinal to $\mathbb{N}$. \\

    Let \( E = \{ 2n \mid n \in \mathbb{N} \} \) be the set of even numbers. We will construct a bijection \( f: \mathbb{N} \to E \) defined by
\[
f(n) = 2n.
\]

This function is both injective and surjective: 

\begin{itemize}
    \item \textit{Injective:} If \( f(n_1) = f(n_2) \), then \( 2n_1 = 2n_2 \) implies \( n_1 = n_2 \).
    \item \textit{Surjective:} For every \( m \in E \), there exists \( n \in \mathbb{N} \) such that \( m = 2n \).
\end{itemize}

Thus, \( E \) is equicardinal to \( \mathbb{N} \).
\end{example}

It is certainly true that $E$ is a proper subset of $\N$, and for this reason it may seem logical to say that $E$ is a smaller set than $\N$. This is one way to look at it, but it represents a point of view that is heavily biased from an overexposure to finite sets.

\begin{example}
    The set of integers is equicardinal to $\mathbb{N}$. \\

    Let \( \mathbb{Z} = \{ \ldots, -2, -1, 0, 1, 2, \ldots \} \). We will construct a bijection \( g: \mathbb{N} \to \mathbb{Z} \) defined by

\[
g(n) = \begin{cases}
\frac{n}{2}, & \text{if } n \text{ is even} \\
-\frac{n+1}{2}, & \text{if } n \text{ is odd}.
\end{cases}
\]

This function is both injective and surjective:

\begin{itemize}
    \item \textit{Injective:} If \( g(n_1) = g(n_2) \), the cases show that \( n_1 = n_2 \).
    \item \textit{Surjective:} For any \( m \in \mathbb{Z} \), there exists \( n \in \mathbb{N} \) such that \( g(n) = m \).
\end{itemize}

Thus, \( \mathbb{Z} \) is equicardinal to \( \mathbb{N} \).
\end{example}

\begin{example}
    The set of all rationals in  $[0, 1]$ \textbf{ is countable.} \\

    To show that the set of all rational numbers in the interval $[0, 1]$ is countable, we consider rational numbers of the form $\frac{p}{q}$, where $q \neq 0$. \\

We start by incrementing $q$ in steps of 1, beginning with $q = 1$. For each integer $q \geq 1$, we consider all integers $p$ such that $0 \leq p \leq q$. The rational number $\frac{p}{q}$ is added to the set if it is not already present. \\

The set of rational numbers in $[0, 1]$ can then be explicitly listed as follows:

\[
\left\{ 0, 1, \frac{1}{2}, \frac{1}{3}, \frac{2}{3}, \frac{1}{4}, \frac{3}{4}, \frac{1}{5}, \frac{2}{5}, \frac{3}{5}, \frac{4}{5}, \frac{1}{6}, \frac{5}{6}, \ldots \right\}
\]

This process demonstrates that we can enumerate all rational numbers in $[0, 1]$. \\ 

Next, we can define a bijection from the set of rational numbers $\mathbb{Q} \cap [0, 1]$ to the natural numbers $\mathbb{N}$. Each rational number $\frac{p}{q}$ is mapped to its index in the above enumeration.\\

Thus, the set of all rational numbers in $[0, 1]$ is countably infinite, and hence it is countable.
\end{example}

\begin{theorem}
    Let \( I \) be a countable index set, and let \( E_i \) be countable for each \( i \in I \). Then \( \bigcup_{i \in I} E_i \) is countable.
\end{theorem}

More glibly, it can also be stated as follows: \textit{A countable union of countable sets is countable.}

\begin{proof}
    Since \( I \) is a countable index set, we can enumerate it as \( I = \{i_1, i_2, i_3, \ldots\} \). For each \( i_j \), since \( E_{i_j} \) is countable, we can enumerate the elements of \( E_{i_j} \) as follows:
\[
E_{i_j} = \{e_{j,1}, e_{j,2}, e_{j,3}, \ldots\}
\]
for \( j = 1, 2, 3, \ldots \).

Now, we can construct a new set \( S \) by listing the elements of the \( E_i \) sets in a systematic way. We can arrange them in a two-dimensional array:
\[
\begin{array}{c|cccc}
    & e_{1,1} & e_{1,2} & e_{1,3} & \cdots \\
    \hline
    E_{i_1} & e_{2,1} & e_{2,2} & e_{2,3} & \cdots \\
    E_{i_2} & e_{3,1} & e_{3,2} & e_{3,3} & \cdots \\
    E_{i_3} & \vdots & \vdots & \vdots & \ddots
\end{array}
\]

We can then define a function \( f : \mathbb{N} \to \bigcup_{i \in I} E_i \) that maps each natural number to a unique element of the union. For instance, we can use the following enumeration method:
\( f(1) = e_{1,1} \), 
\( f(2) = e_{1,2} \), 
\( f(3) = e_{2,1} \), 
\( f(4) = e_{1,3} \), 
\( f(5) = e_{2,2} \), 
\( f(6) = e_{3,1} \), 
and so on. \\

This enumeration process ensures that every element in the union \( \bigcup_{i \in I} E_i \) will eventually be listed. Therefore, we can conclude that \( \bigcup_{i \in I} E_i \) is countable.

\end{proof}

\begin{example}
    The set of all Rational numbers, $\Q$ is countable. \\

    We will now use theorem \textbf{1.5} to prove the countability of the set of all rational numbers. \\
    
    It has been already proved that the set \( \mathbb{Q} \cap [0, 1] \) is countable. Similarly, it can be shown that \( \mathbb{Q} \cap [n, n + 1] \) is countable, \( \forall n \in \mathbb{Z} \). Let \( Q_i = \mathbb{Q} \cap [i, i + 1] \). Thus, clearly, the set of all rational numbers, \( \mathbb{Q} = \bigcup_{i \in \mathbb{Z}} Q_i \) – a countable union of countable sets – is countable. \\

\end{example}

\begin{definition}
    A set $F$ is uncountable if it has cardinality strictly greater than the cardinality of $\N$. \\
\end{definition}

\begin{theorem}
    The set of all infinite binary strings, $\{0, 1\}^\infty$, is uncountable.
\end{theorem}

\begin{proof}
    Assume for the sake of contradiction that the set of all binary strings, $A = \{0, 1\}^\infty$, is countably infinite. Thus, there exists a bijection $f: A \to \mathbb{N}$. In other words, we can order the set of all infinite binary strings as follows: 

\[
\begin{array}{cccc}
a_{11} & a_{12} & a_{13} & \ldots \\
a_{21} & a_{22} & a_{23} & \ldots \\
a_{31} & a_{32} & a_{33} & \ldots \\
\vdots & \vdots & \vdots & \ddots \\
\end{array}
\]

where $a_{ij}$ is the $j$th bit of the $i$th binary string, $i, j \geq 1$. \\

Consider the infinite binary string given by $\bar{a} = \bar{a}_{11} \bar{a}_{22} \bar{a}_{33} \ldots$, where $\bar{a}_{ij}$ is the complement of the bit $a_{ij}$. \\

Since our list contains all infinite binary strings, there must exist some $k \in \mathbb{N}$ such that the string $\bar{a}$ occurs at the $k$th position in the list, i.e., $f(\bar{a}) = k$. The $k$th bit of this specific string is $\bar{a}_{kk}$. However, from the above list, we know that the $k$th bit of the $k$th string is $a_{kk}$. Thus, we can conclude that the string $\bar{a}$ cannot occur in any position $k \geq 1$ in our list, contradicting our initial assumption that our list exhausts all possible infinite binary strings. \\

Thus, there cannot possibly exist a bijection from $\mathbb{N}$ to $\{0, 1\}^\infty$, proving that $\{0, 1\}^\infty$ is uncountable.\\
\end{proof}


\begin{corollary}
    The sets $[0, 1]$, $\R$ and $\{\R \setminus \Q\}$ are uncountable.
\end{corollary}

\begin{proof}
    Firstly, consider the set $[0, 1]$. Any number in this set can be expressed by its binary equivalent, which suggests a bijection from $[0, 1]$ to $\{0, 1\}^\infty$. However, this is not exactly a bijection due to the issue with the dyadic rationals (i.e., numbers of the form $\frac{a}{2^b}$, where $a$ and $b$ are natural numbers, and $a$ is odd). For example, $0.01000\ldots$ in binary is the same as $0.001111\ldots$. \\ 

To address this, we can tweak this "near bijection" to produce an explicit bijection as follows. For any infinite binary string $x = (x_1, x_2, \ldots) \in \{0, 1\}^\infty$, let
\[
g(x) = \sum_{k=1}^{\infty} x_k 2^{-k}.
\]
The function $g$ maps $\{0, 1\}^\infty$ "almost bijectively" to $[0, 1]$, but the dyadic rationals have two pre-images. For instance, we have $g(1000\ldots) = g(0111\ldots) = \frac{1}{2}$. \\

To resolve this, let the set of dyadic rationals be given by the list
\[
D = \left\{ 
d_1 = \frac{1}{2}, d_2 = \frac{1}{4}, d_3 = \frac{3}{4}, d_4 = \frac{1}{8}, d_5 = \frac{3}{8}, d_6 = \frac{5}{8}, d_7 = \frac{7}{8}, \ldots 
\right\}.
\]

Note that the dyadic rationals can be enumerated as they are countable. \\

Next, we define the following bijection $f(x)$ from $\{0, 1\}^\infty$ to $[0, 1]$:

\[
f(x) =
\begin{cases}
g(x) & \text{if } g(x) \notin D, \\
d_{2n-1} & \text{if } g(x) = d_n \text{ for some } n \in \mathbb{N} \text{ and } x_k \text{ terminates in } 1, \\
d_{2n} & \text{if } g(x) = d_n \text{ for some } n \in \mathbb{N} \text{ and } x_k \text{ terminates in } 0.
\end{cases}
\]

This defines an explicit bijection from $\{0, 1\}^\infty$ to $[0, 1]$, proving that the set $[0, 1]$ is uncountable. \\

Next, we can define a bijection from $(0, 1)$ to $\mathbb{R}$, for example using the function $\tan\left(\pi x - \frac{\pi}{2}\right)$ for $x \in (0, 1)$. Thus, the set of all real numbers, $\mathbb{R}$, is uncountable. \\

Finally, we write $\mathbb{R} = \mathbb{Q} \cup (\mathbb{R} \setminus \mathbb{Q})$. Since $\mathbb{Q}$ is countable and $\mathbb{R}$ is uncountable, we conclude that $\mathbb{R} \setminus \mathbb{Q}$, the set of all irrational numbers, is also uncountable. \\

\end{proof}


\begin{example}
    Prove that \( \mathcal{P}(\mathbb{N}) \), the power set of the natural numbers, is uncountable. \\
    
    To do so, we can use the method of contradiction and Cantor's diagonal argument. \\

    Let us assume that \( \mathcal{P}(\mathbb{N}) \) is countable. This means that we can list all the subsets of \( \mathbb{N} \) as follows:
    
    \[
    S_1, S_2, S_3, \ldots
    \]
    
    Next, we will associate each subset \( S_i \) with an infinite binary string \( b_i \), where the \( n \)-th digit of \( b_i \) is defined as follows:
    
    \[
    b_i(n) = 
    \begin{cases} 
    1 & \text{if } n \in S_i \\ 
    0 & \text{if } n \notin S_i 
    \end{cases}
    \]
    
    Thus, each subset \( S_i \) corresponds to a binary string that represents whether each natural number is included in the subset or not. \\
    
    Now, consider the set \( S \) of all natural numbers that belong to none of the subsets listed. Specifically, we define \( S \) as follows:
    
    \[
    S = \{ n \in \mathbb{N} : n \notin S_n \}
    \]
    
    By this definition, for each \( n \), the \( n \)-th digit of the binary string corresponding to \( S \) will be: \\
    
    \[
    b(n) = 
    \begin{cases} 
    1 & \text{if } n \in S \\ 
    0 & \text{if } n \notin S 
    \end{cases}
    \]
    
    Now, we need to determine whether \( S \) is included in our original list of subsets \( S_1, S_2, S_3, \ldots \). \\

    If \( S = S_k \) for some \( k \), then by the construction of \( S \), we have:
    \begin{itemize}
        \item If \( k \in S \), then by definition of \( S \), \( k \notin S_k \), which is a contradiction.
        \item  Conversely, if \( k \notin S \), then \( k \in S_k \), which again leads to a contradiction.
    \end{itemize}
    
    Since \( S \) cannot be equal to any \( S_k \) in our assumed countable list, this means \( S \) is a subset of \( \mathbb{N} \) that is not included in our original enumeration of subsets. \\
    
    Thus, we arrive at a contradiction. Therefore, our initial assumption that \( \mathcal{P}(\mathbb{N}) \) is countable must be false, and we conclude that:
    
    \[
    \mathcal{P}(\mathbb{N}) \text{ is uncountable.}
    \]    
\end{example}

\begin{example}
    Show that an infinite subset of a countable set is countable. \\

    Let $S$ be countably infinite set. So, there exists a bijection \( f: \mathbb{N} \to S \), meaning we can enumerate the elements of \( S \) as \( s_1, s_2, s_3, \ldots \). \\

    Assuming \( A \) is an infinite subset of \( S \), we can construct a sequence of elements from \( A \):
    \begin{itemize}
        \item Start with the first element \( a_1 \in A \).
        \item Find the next element \( a_2 \in A \) such that \( a_2 \) is greater than \( a_1 \) in the enumeration of \( S \).
        \item  Continue this process to find \( a_3, a_4, \ldots \), ensuring each \( a_n \) is greater than \( a_{n-1} \).
    \end{itemize}

    Since \( A \) is infinite, this process will yield an infinite sequence \( a_1, a_2, a_3, \ldots \) where each \( a_n \) is an element of \( A \).\\

We have thus constructed a function \( g: \mathbb{N} \to A \) that enumerates the elements of \( A \), showing that \( A \) is countable.
\end{example}
